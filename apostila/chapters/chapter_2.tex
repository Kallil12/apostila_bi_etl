\chapter{Important ETL Concepts}

In the context of IT, ETL stands for Extract, Transform, Load, which is a process used to integrate and consolidate data from various sources into a target system or data warehouse. ETL plays a crucial role in data management and enables organizations to extract valuable insights and make informed business decisions.

The first step in the ETL process is extraction, where data is collected from different sources such as databases, files, APIs, or external systems. This involves retrieving relevant data based on defined criteria and requirements. Once the data is extracted, it undergoes transformation, which involves cleaning, validating, and structuring the data to ensure consistency, accuracy, and compatibility with the target system. Transformations may include data cleansing, data enrichment, data aggregation, and applying business rules or calculations.

Finally, the transformed data is loaded into the target system, typically a data warehouse or a data mart. This step involves mapping the transformed data to the appropriate tables or entities in the target system, ensuring data integrity and maintaining the overall data structure. The loaded data can then be used for reporting, analysis, and decision-making purposes.

\section{Slow Changing Dimensions (SCDs)}

Slow Changing Dimensions refer to the handling and management of slowly changing data in a data warehousing environment. In ETL processes, SCDs are used to track and manage changes to dimensional attributes over time.

SCDs are essential because they enable the capture and representation of historical data in a data warehouse. This is particularly useful when analyzing trends, performing historical comparisons, or conducting time-series analysis.

There are different types of Slow Changing Dimensions, including:

\begin{itemize}
	\item Type 1 SCD: Overwrite the existing dimension attribute with the new value, effectively losing the historical information. This approach is suitable when historical data is not needed or when the dimension attribute is not expected to change over time.
	
	\item Type 2 SCD: Maintain a separate row for each change in the dimension attribute, creating a new record with an updated attribute value and an assigned surrogate key. This approach allows for historical tracking but can lead to data redundancy.
	
	\item Type 3 SCD: Add new columns to the dimension table to track specific changes, typically storing the current value and previous value of the attribute. This approach offers limited historical tracking but can help maintain simplicity in the data model.
	
	\item Type 4 SCD: Create a separate history table to store all changes to the dimension attribute, while the main dimension table only contains the current value. This approach provides comprehensive historical tracking while minimizing redundancy in the main dimension table.
\end{itemize}